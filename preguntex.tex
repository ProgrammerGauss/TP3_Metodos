preguntar que valores puede tomar sigma
que serian p dimensiones
que serian las respuestas o etiquetas


2. 

SVD
Idea: 2 conjuntos de vectores singulares {ui} y {vi} 
A de R^{m x m} de rango r (#filas o col LI)
Obs: r <= min(m,n)
A tiene r valores singulares no negativos {sigma_i} en orden decreciente


n = 19
p*p aprox = 700

=> A es gorda y petiza

importar imagenes con imageioV2
definir una array A = np.zeros((n,n,p))
for i in n:
    imagen.flatten()

recuperar tamanio de imagen
np.reshape(A[i, i], img.shape)

graficar (imshow)
    v sub i con i = 1 a 19

en este caso U va a ser el numero que corresponde a cada imagen
    vector que tiene 19 elementos
-> me va a dar la mejor aproximacion de la imagen de rango R

u

np.SVD(A)
U, sigma, V = np.SVD(A)
A = U @ np.diag(sigma) @ V.T

USAR EL TEOREMA DE ECKART Y YOUNG



Compresion
si tengo U S y V de A
A_k = U[:, :k] @ S[:k, :k] @ V[:k, :]
ESTA ES LA VERSION REDUCIDA 

Matrices autosimilares
    Matriz de cosenos:
        